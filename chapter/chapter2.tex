\setcounter{section}{1}
\section{Introduction to quantum mechanics}
\subsection{}
\begin{align}
	\begin{bmatrix}
		1 \\ 
		-1
	\end{bmatrix} 
	+
	\begin{bmatrix}
		1 \\ 
		2
	\end{bmatrix}
	-
	\begin{bmatrix}
		2 \\ 
		1
	\end{bmatrix}
	=
	\begin{bmatrix}
		0 \\ 
		0
	\end{bmatrix}	
\end{align}

\subsection{}
\begin{align}
	A\ket{0} &= A_{11}\ket{0} + A_{21}\ket{1} = \ket{1} \Rightarrow A_{11} = 0,\ A_{21} = 1\\
	A\ket{1} &= A_{12}\ket{0} + A_{22}\ket{1} = \ket{0} \Rightarrow A_{12} = 1,\ A_{22} = 0\\
%
	\therefore A &= 	
	\begin{bmatrix}
		0 & 1 \\ 
		1 & 0
	\end{bmatrix} 
\end{align}\\

\begin{align}
	A &= 	
	\begin{bmatrix}
	1 & 0 \\ 
	0 & 1
	\end{bmatrix} \text{ w.r.t. } \left\{\ket{1},\ \ket{0} \right\}
\end{align}


\subsection{}
\begin{align}
	A \ket{v_i} &= \sum_{j} A_{ji}\ket{w_j}\\
	B \ket{w_j} &= \sum_{k} B_{kj}\ket{x_k}
\end{align}
%
Thus
\begin{align}
	BA \ket{v_i} &= B \left( \sum_{j} A_{ji}\ket{w_j} \right)\\
	&= \sum_{j} A_{ji} B\ket{w_j}\\
	&= \sum_{j,k} A_{ji} B_{kj}\ket{x_k}\\
	&= \sum_k \left( \sum_j B_{kj} A_{ji}  \right) \ket{x_k}\\
	&= \sum_k (BA)_{ki} \ket{x_k}
\end{align}



\subsection{}
\begin{align}
	I\ket{v_j} = \sum_i I_{ij} \ket{v_i} = \ket{v_j},\ \forall j.\\
	\Rightarrow I_{ij} = \delta_{ij}
\end{align}

\setcounter{subsection}{5}
\subsection{}
\begin{align}
	\left(\sum_i \lambda_i \ket{w_i},\ \ket{v}\right) &=
	\left(\ket{v},\ \sum_i \lambda_i \ket{w_i}\right)^*\\
	&= \left[\sum_i \lambda_i \left(\ket{v},\ \ket{w_i}  \right) \right]^*\\
	&= \sum_i \lambda_i^* \left(\ket{v},\ \ket{w_i} \right)^*\\
	&= \sum_i \lambda_i^* (\ket{w_i},\ \ket{v})
\end{align}


\subsection{}
\begin{align}
	\braket{w | v} &= \begin{bmatrix}
		1 & 1
	\end{bmatrix} 
	\begin{bmatrix}
	1 \\ 
	-1
	\end{bmatrix} 
	= 1 - 1 = 0\\
%	
	\frac{\ket{w}}{\norm{\ket{w}}} &= 
	\frac{\ket{w}}{\sqrt{\braket{w|w}}} = \frac{1}{\sqrt{2}} \begin{bmatrix}
	1 \\ 
	1
	\end{bmatrix}\\
%	
	\frac{\ket{v}}{\norm{\ket{v}}} &= 
	\frac{\ket{v}}{\sqrt{\braket{v|v}}} = \frac{1}{\sqrt{2}} \begin{bmatrix}
	1 \\ 
	-1
	\end{bmatrix}
\end{align}


\setcounter{subsection}{8}
\subsection{}
\begin{align}
	\sigma_0 &= I = \ket{0}\bra{0} + \ket{1}\bra{1}\\
	\sigma_1 &= X = \ket{0}\bra{1} + \ket{1}\bra{0}\\
	\sigma_2 &= Y = -i\ket{0}\bra{1} + i\ket{1}\bra{0}\\
	\sigma_3 &= Z = \ket{0}\bra{0} - \ket{1}\bra{1}
\end{align}


\subsection{}
\begin{align}
	\ket{v_j}\bra{v_k} &= I_V \ket{v_j} \bra{v_k} I_V\\
	&= \left(\sum_p \ket{v_p}\bra{v_p} \right) \ket{v_j}\bra{v_k} \left(\sum_q \ket{v_q}\bra{v_q} \right)\\
	&= \sum_{p,q} \ket{v_p} \braket{v_p|v_j}
	\braket{v_k | v_q} \bra{v_q}\\
	&= \sum_{p,q} \delta_{pj} \delta_{kq} \ket{v_p} \bra{v_q}
\end{align}
Thus
\begin{align}
	\left( \ket{v_j}\bra{v_k} \right)_{pq} = \delta_{pj} \delta_{kq}
\end{align}



\subsection{}
\begin{align}
	X = \begin{bmatrix}
	0 & 1 \\ 
	1 & 0
	\end{bmatrix},\ \det(X-\lambda I) = 
	\det \left(\begin{bmatrix}
	-\lambda & 1 \\ 
	1 & -\lambda
	\end{bmatrix} \right) = 0 \Rightarrow \lambda \pm 1
\end{align}

If $\lambda = -1$,
\begin{align}
	\begin{bmatrix}
		1 & 1 \\ 
		1 & 1
	\end{bmatrix} 
	\begin{bmatrix}
		c_1 \\ 
		c_2 
	\end{bmatrix} = 
	\begin{bmatrix}
		0 \\ 
		0 
	\end{bmatrix}
\end{align}
Thus
\begin{align}
	\ket{\lambda = -1} = \begin{bmatrix}
	c_1 \\ 
	c_2 
	\end{bmatrix} = \frac{1}{\sqrt{2}} 
	\begin{bmatrix}
	-1 \\ 
	1 
	\end{bmatrix}
\end{align}

If $\lambda = 1$
\begin{align}
	\ket{\lambda = 1} = \frac{1}{\sqrt{2}} 
	\begin{bmatrix}
	1 \\ 
	1 
	\end{bmatrix}
\end{align}

\begin{align}
	X = \begin{bmatrix}
	-1 & 0 \\ 
	0 & 1
	\end{bmatrix} 
	\text{ w.r.t. } \left\{ \ket{\lambda = -1},\ \ket{\lambda = 1}\right\}
\end{align}



\subsection{}
\begin{align}
	\det \left(\begin{bmatrix}
	1 & 0 \\ 
	1 & 1
	\end{bmatrix} - \lambda I \right) = (1 - \lambda)^2 = 0 \Rightarrow \lambda = 1
\end{align}
Therefore the eigenvector associated with eigenvalue $\lambda = 1$ is 
\begin{align}
	\ket{\lambda = 1} = \begin{bmatrix}
	0 \\ 
	1
	\end{bmatrix} 
\end{align}

Because $\ket{\lambda = 1}\bra{\lambda = 1} = \begin{bmatrix}
0 & 0 \\ 
0 & 1
\end{bmatrix}$, 
\begin{align}
	\begin{bmatrix}
	1 & 0 \\ 
	1 & 1
	\end{bmatrix} \neq c\ket{\lambda = 1}\bra{\lambda = 1} = \begin{bmatrix}
	0 & 0 \\ 
	0 & c
	\end{bmatrix}
\end{align}



\subsection{}
Suppose $\ket{\psi},\ \ket{\phi}$ are arbitrary vectors in $V$.
\begin{align}
	\left(\ket{\psi},\ (\ket{w}\bra{v}) \ket{\phi}\right)^* &=
	\left((\ket{w}\bra{v})^\dagger \ket{\psi},\  \ket{\phi}\right)^*\\
	&= \left(\ket{\phi},\ (\ket{w}\bra{v})^\dagger \ket{\psi} \right)\\
	&= \bra{\phi} (\ket{w}\bra{v})^\dagger \ket{\psi}.
\end{align}

On the other hand,
\begin{align}
	\left(\ket{\psi},\ (\ket{w}\bra{v}) \ket{\phi}\right)^* 
	&= (\braket{\psi | w} \braket{v | \phi})^*\\
	&= \braket{\phi | v} \braket{w | \psi}.
\end{align}

Thus
\begin{align}
	\bra{\phi} (\ket{w}\bra{v})^\dagger \ket{\psi} = \braket{\phi | v} \braket{w | \psi} \text{ for arbitrary vectors } \ket{\psi},\ \ket{\phi}\\
	\therefore (\ket{w}\bra{v})^\dagger = \ket{v}\bra{w}
\end{align}


\subsection{}
\begin{align}
	( (a_i A_i)^\dagger \ket{\phi},\ \ket{\psi} )
	&= (\ket{\phi},\ a_i A_i \ket{\psi})\\
	&= a_i (\ket{\phi},\ A_i \ket{\psi})\\
	&= a_i (A_i^\dagger \ket{\phi},\ \ket{\psi})\\
	&= (a_i^* A_i^\dagger \ket{\phi},\ \ket{\psi})\\
%	
	\therefore (a_i A_i)^\dagger = a_i^* A_i^\dagger
\end{align}




\subsection{}
\begin{align}
	((A^\dagger)^\dagger\ket{\psi},\ \ket{\phi} )
	&= (\ket{\psi},\ A^\dagger \ket{\phi})\\
	&= (A^\dagger \ket{\phi},\ \ket{\psi})^*\\
	&= (\ket{\phi},\ A\ket{\psi})^*\\
	&= (A\ket{\psi},\ \ket{\phi})\\
	\therefore (A^\dagger)^\dagger = A
\end{align}


\subsection{}
\begin{align}
	P &= \sum_i \ket{i}\bra{i}.\\
	P^2 &= \left(\sum_i \ket{i}\bra{i}\right) \left(\sum_j \ket{j}\bra{j}\right)\\
	&= \sum_{i,j} \ket{i}\braket{i | j}\bra{j}\\
	&= \sum_i \ket{i}\bra{j} \delta_{ij}\\
	&= \sum_i \ket{i}\bra{i}\\
	&= P
\end{align}




\setcounter{subsection}{17}
\subsection{}
Suppose $\ket{v}$ is a eigenvector with corresponding eigenvalue $\lambda$.
\begin{align}
	U \ket{v} &= \lambda \ket{v}.\\
	1 &= \braket{v | v}\\
	&= \bra{v} I \ket{v}\\
	&= \bra{v} U^\dagger U \ket{v}\\
	&= \lambda \lambda^* \braket{v | v}\\
	&= \norm{\lambda}^2\\
	\therefore \lambda &= e^{i \theta}
\end{align}




\subsection{}
\begin{align}
	X^2 = \begin{bmatrix}
		0 & 1 \\ 
		1 & 0
	\end{bmatrix} 
	\begin{bmatrix}
		0 & 1 \\ 
		1 & 0
	\end{bmatrix}
	= \begin{bmatrix}
		1 & 0 \\ 
		0 & 1
	\end{bmatrix} = I
\end{align}



\subsection{}
\begin{align}
	U &\equiv \sum_i \ket{w_i}\bra{v_i}\\
	A_{ij}^{'} &= \braket{v_i | A | v_j}\\
	&= \braket{v_i | UU^\dagger A UU^\dagger | v_j}\\
	&= \sum_{p,q,r,s} \braket{v_i | w_p} \braket{v_p | v_q} \braket{w_q | A | w_r} \braket{v_r | v_s} \braket{w_s | v_j}\\
	&= \sum_{p,q,r,s} \braket{v_i | w_p} \delta_{pq} A_{qr}^{''} \delta_{rs}  \braket{w_s | v_j}\\
	&= \sum_{p,r}  \braket{v_i | w_p}  \braket{w_r | v_j} A_{pr}^{''}
\end{align}


\setcounter{subsection}{25}
\subsection{}





